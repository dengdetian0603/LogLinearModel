\documentclass[]{beamer}
% Class options include: notes, notesonly, handout, trans,
%                        hidesubsections, shadesubsections,
%                        inrow, blue, red, grey, brown

% Theme for beamer presentation.
\usepackage{beamerthemesplit} 
\usefonttheme{professionalfonts}

\newcommand\Fontvi{\fontsize{7}{7}\selectfont}
\newcommand\Fontvii{\fontsize{8}{8}\selectfont}
\newcommand{\argmax}[1]{\underset{#1}{\operatorname{arg}\,\operatorname{max}}\;}
% Other themes include: beamerthemebars, beamerthemelined, 
%                       beamerthemetree, beamerthemetreebars  



\title{Log Linear Models and Multivariate Binary Data }    % Enter your title between curly braces
\author{Detian Deng}                 % Enter your name between curly braces
\institute{Department of Biostatistics \\Johns Hopkins Bloomberg School of Public Health}      % Enter your institute name between curly braces
\date{\today}                    % Enter the date or \today between curly braces

\begin{document}

% Creates title page of slide show using above information
\begin{frame}
  \titlepage
\end{frame}
%\note{Talk for 30 minutes} % Add notes to yourself that will be displayed when
                           % typeset with the notes or notesonly class options

\begin{frame}
\frametitle{Reference}
\Fontvi
\begin{itemize}
\item Cox, D. R. (1972). The analysis of multivariate binary data. Applied statistics, 113-120.
\item Zhao, L. P., \& Prentice, R. L. (1990). Correlated binary regression using a quadratic exponential model. Biometrika, 77(3), 642-648.
\item Fitzmaurice, G. M., \& Laird, N. M. (1993). A likelihood-based method for analysing longitudinal binary responses. Biometrika, 80(1), 141-151.
\item Fitzmaurice, G. M., Laird, N. M., \& Rotnitzky, A. G. (1993). Regression models for discrete longitudinal responses. Statistical Science, 284-299.
\item Molenberghs, G., \& Ritter, L. L. (1996). Methods for analyzing multivariate binary data, with association between outcomes of interest. Biometrics, 1121-1133.
\item Dai, B., Ding, S., \& Wahba, G. (2013). Multivariate Bernoulli distribution. Bernoulli, 19(4), 1465-1483.
\end{itemize}

\end{frame}

\section[Outline]{}

% Creates table of contents slide incorporating
% all \section and \subsection commands
\begin{frame}
  \tableofcontents
\end{frame}
%
%\begin{frame}
%  \frametitle{Notations}   % Insert frame title between curly braces
%
%  \begin{itemize}
%\item $X_1, X_2, \ldots, X_N$ are $N$ independent observations on $X$
%  \end{itemize}
%\end{frame}


\section{Log Linear Representation}

\subsection{Cox (1972)}
\begin{frame}
  \frametitle{Cox (1972)}   % Insert frame title between curly braces
Suppose $Y$ is a K-dimensional multivariate Bernoulli random variable.
\begin{align}
f(y_i, \Psi_i, \Omega_i) = & \exp \{\Psi_i^T y_i + \Omega_i^T w_i - A(\Psi_i,\Omega_i)\}
\end{align}
where $W_i$ is a $(2^K - K - 1) \times 1$ vector of two- and  higher- way cross-products of $Y_i$, $\Psi_i, \Omega_i$ are vectors of canonical parameters (conditional log odds ratios), and $A(\Psi_i,\Omega_i)$ is a normalizing constant, $\exp \{A(\Psi_i,\Omega_i)\} = \sum \exp (\Psi_i^T y_i + \Omega_i^T w_i)$, summing over all $2^K$ possible values of $Y_i$.
\end{frame}

\subsection{Zhao and Prentice (1990)}
\begin{frame}
  \frametitle{Zhao and Prentice (1990)}   % Insert frame title between curly braces

Quadratic Exponential Family, or pairwise model:
\begin{align}
f(y_i, \Psi_i, \Omega_i) = & \exp \{\Psi_i^T y_i + \Omega_i^{T} u_i - A(\Psi_i,\Omega_i)\}
\end{align}
where the three- and higher- way association parameters are set to zero, i.e.
$U_i = (Y_{i1}Y_{i2}, \ldots, Y_{iK-1}Y_{iK})$ is a $K(K-1)/2 \times 1$ vector of two-way cross-products.\\
\ \\
A one-to-one transformation from $(\Psi_i, \Omega_i)$ to the moment parameters $(\mu_i,\sigma_i)$ was made in order to obtain the likelihood equations, where $\sigma_i$ is the vector of marginal correlations.
\end{frame}


\subsection{Fitzmaurice and Laird (1993)}

\begin{frame}
\frametitle{Fitzmaurice and Laird (1993)}
``Mixed parameter" model based on the general log-linear representation $(1)$.\\
\ \\
A model for the mean, $\mu_i$, and the canonical association parameters, $\Omega_i$, is assumed and likelihood equations are obtained via the 1-1 transformation from 
$(\Psi_i, \Omega_i)$ to $(\mu_i, \Omega_i)$, although there is , in general, no closed form expression representing the joint probabilities as a function of $\mu_i$ and $\Omega_i$.
\end{frame}

%\subsection{Molenberghs and Ritter (1996)}
%
%\begin{frame}
%\frametitle{Molenberghs and Ritter (1996)}
%Re-partition the general log-linear model: 
%\begin{align}
%f(y_i, \Psi_i, \Omega_i) = & \exp \{\Psi_i^T v_i + \Omega_i^T w_i - A(\Psi_i,\Omega_i)\}
%\end{align}
%where $V_i$ contains the outcomes and pairwise cross-products, $W_i$ contains the three- and higher- way cross-products
%
%\end{frame}





\section{Conditional Multivariate Bernoulli Distribution}
\subsection{Pairwise Model Parameterization}

\begin{frame}
\frametitle{Pairwise Model Parameterization}
\Fontvii
\begin{align*}
P(Y_i=y|\sum_{j=1}^K Y_{ij} = s) = & \frac{P(Y_i=y,\sum_{j=1}^K Y_{ij} = s)}{P(\sum_{j=1}^K Y_{ij} = s)}\\
= & \frac{1(\sum_{j=1}^K y_j = s) P(Y_i=y)}{P(\sum_{j=1}^K Y_{ij} = s)} \\
= & \frac{1(\sum_{j=1}^K y_j = s)\exp \{\Psi_i^T y + \Omega_i^{T} u - A(\Psi_i,\Omega_i) \}}{\sum \exp \{\Psi_i^T y + \Omega_i^{T} u - A(\Psi_i,\Omega_i)\}}\\
= & 1(\sum_{j=1}^K y_j = s)\exp \{\Psi_i^T y + \Omega_i^{T} u - B(\Psi_i,\Omega_i,s) \}
\end{align*}
where $B(\Psi_i,\Omega_i, s)$ is a normalizing constant, $\exp \{B(\Psi_i,\Omega_i,s)\} = \sum \exp (\Psi_i^T y + \Omega_i^T u)$, summing over all ${K \choose s}$ possible values of $Y$.

\end{frame}

\subsection{Parameter Estimation}
\begin{frame}
\frametitle{Estimation}

\end{frame}


\end{document}


%\subsection{Theorem 3}
%\subsection{Decomposition of Fisher' s Information}

%\begin{frame}
  %\frametitle{Decomposition of Fisher' s Information}   % Insert frame title between curly braces

 % \begin{itemize}
  %\item<1-> Point 1 (Click ``Next Page'' to see Point 2) % Use Next Page to go to Point 2
  %\item<2-> Point 2  % Use Next Page to go to Point 3
  %\item<3-> Point 3
  %\end{itemize}
%\end{frame}
%\note{Speak clearly}  % Add notes to yourself that will be displayed when
                      % typeset with the notes or notesonly class options


%\section{Decomposition of Fisher' s Information}

%\begin{frame}
  %\frametitle{Decomposition of Fisher' s Information}   % Insert frame title between curly braces
  %\begin{columns}[c]
  %\column{2in}  % slides are 3in high by 5in wide
  %\begin{itemize}
  %\item<1-> First item
  %\item<2-> Second item
  %\item<3-> ...
  %\end{itemize}
 % \column{2in}
  %\framebox{Insert graphic here % e.g. \includegraphics[height=2.65in]{graphic}
  %}
  %\end{columns}
%\end{frame}
%\note{The end}       % Add notes to yourself that will be displayed when
		     % typeset with the notes or notesonly class options


