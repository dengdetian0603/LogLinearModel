\documentclass[11 pt, a4paper]{article}  % list options between brackets
\usepackage[nottoc,notlof,notlot,numbib]{tocbibind}
\usepackage[margin=1.1 in]{geometry}
\usepackage{fancyhdr}
\usepackage{manfnt}
\usepackage{pgf}
\usepackage{amsmath,amssymb,natbib,graphicx}
\usepackage{amsfonts}
\DeclareMathAlphabet{\mathpzc}{OT1}{pzc}{m}{it}
\usepackage{bbm}
\usepackage{hyperref}
\usepackage{float}
\usepackage{mathrsfs} %mathscr{A}
\usepackage{subfigure}
\usepackage{chngpage}

\usepackage{times}
\usepackage{latexsym}
\usepackage{caption}

\usepackage{algorithm}
\usepackage[noend]{algpseudocode}

\newtheorem{axiom}{Axiom}[section]
\newtheorem{result}{Result}[section]
\newtheorem{example}{example}[section]
\newtheorem{definition}{Definition}[section]
\newtheorem{principle}{Principle}[section]
\newtheorem{theorem}{Theorem}[section]% list packages between braces

% type user-defined commands here
%\newcommand{\vmu}{{\bf \mu}}
%\newcommand{\vtheta1}{{\bf \theta^{(1)}}}
%\newcommand{\vtheta2}{{\bf \theta^{(2)}}}
%\newcommand{\vpi}{{\bf \pi}}}

\newcommand{\gm}{\gamma}

\begin{document}

\title{Summary of Short-term Research Objectives}   % type title between braces
\author{Detian Deng}         % type author(s) between braces
\date{\today}    % type date between braces
\maketitle

%\begin{abstract}
%\end{abstract}

\section{Model Specification}             % section 1
Let $L$ be a K-dimensional Bernoulli random variable denoting the true state.
Consider the general log linear model:
\begin{align*}
f(l; \Theta) = & \exp \{\Theta_1^T l + \Theta_2^{T} u_2 + \ldots + \Theta_K^T u_K - A^*(\Theta)\}
\end{align*}
where $U_k$ is a ${K \choose k} \times 1$ vector of k-way cross-products, $k = 1,\ldots,K$,  and $\Theta = (\Theta_1,\ldots, \Theta_K)$ contains the the natural parameters, which is a $(2^K-1) \times 1$ vector.\\
\ \\
Model restrictions, let $\tilde{l} = (l,u_2,\dots,u_K)^T$, and $S = \sum_{j=1}^K L_j = s$ has some fixed pmf 
\begin{align}
\pi(s) := & P(S=s)\nonumber \\ 
= & \frac{1}{A(\Theta)} \sum_{\tilde{l}:S=s}\exp \{ \Theta^T \tilde{l}\}  \text{ , } s = 0,1,\ldots, K \\
A(\Theta) = & \sum_{\tilde{l}:l\in \{0,1\}^K}\exp \{ \Theta^T \tilde{l}\}
\end{align}


\newpage
\section{Posterior Distribution}

\subsection{General Model}
For General model, $\tilde{L}$ is a square matrix with dimension $J_1 = 2^K-1$. Recall that 
\begin{align}
A(\Theta) = & \frac{1}{\pi(0)} \nonumber \\
\pi(s) = & \frac{1}{A(\Theta)} \sum_{\tilde{l}:S=s}\exp \{ \Theta^T \tilde{l}\}  \text{ , } s = 1,\ldots, K \\
\mu_k = & \frac{1}{A(\Theta)} \sum_{\tilde{l}:l_k=1}\exp \{ \Theta^T \tilde{l}\}  \text{ , } k = 1, \ldots, K
\end{align}
Define intermediate parameter $\phi_j = \exp(\theta^T\tilde{l}_j) >0$, $\Theta = (\theta^{(1)},\theta^{(2)}, \ldots, \theta^{(K)}), j= 1, \ldots, J_1$ and two $K \times J_1$ sub-design matrices $B$, $C$, where $B[k,j] = 1(\sum_{s=1}^K\tilde{L}[j,s]=k)$, $C[k,j] = \tilde{L}[j,k]$. Thus (3) and (4) become
\begin{align}
\vec{\phi} > & 0\\
B \vec{\phi} = & \vec{\pi}/\pi(0)\\ 
C \vec{\phi} = & \vec{\mu}/\pi(0)
\end{align}
Note that B and C are not independent constraints and should be compatible so that $\binom{B}{C}$ has rank $2K-1$.
Explicitly, $\mu$ and $\pi$ must satisfy  
\[ \sum_{k=1}^K \mu_k = \sum_{k=1}^K k\pi_k\]
Based on [1][2], we can sample $\vec{\phi}$ from Uniform distribution subject to the above linear constraints efficiently and robustly. Then $\Theta$ are the solutions to the linear system ($J_1$ equations with $J_1$ unknowns):
\[\tilde{L}\Theta = \log \vec{\phi}\]

Then the posterior distribution of $(\mu, \pi)$ given data $L$ becomes:
\begin{align*}
P(\mu, \pi |L) \propto & P(L, \mu, \pi) \\
= & P(L | \mu, \pi) P(\mu, \pi) \\
= & \int P(L, \phi | \mu, \pi) d \phi P(\mu, \pi) \\
\propto & \int P(L | \phi) P(\phi | \mu,\pi) d \phi P(\mu, \pi) \\
\approx & \sum P(L | \phi_h) P(\pi | \mu) P(\mu)
\end{align*}
where $P(L | \phi_h)$ is the log linear model density evaluated at $\phi_h$, which is sampled uniformly from the feasible region subject to constrains (5-7).\\
$P(\mu)$ is a multivariate logit Normal density with mean $0$ and standard deviation $\Sigma$.\\
$P(\pi | \mu) = P(\pi_0 | \mu)P(\pi_1,.., \pi_K |\pi_0, \alpha)$, where $P(\pi_0 | \mu)$ is a uniform density on $(0, 1-\max(\mu))$ and $P(\pi_1,..\pi_K |\pi_0, \alpha)$ is a Stick Breaking prior with shape parameter $\alpha$ on $(0, 1-\pi_0)$.


%\subsubsection{QE model}
%For QE model, $\tilde{L}$ is a $J_1\times J_2$, where $J_1 = 2^K-1, J_2 = \frac{1}{2}K(K+1)$.  Let $\theta = (\theta^{(1)},\theta^{(2)})$, we have a over-determined linear system after sampling $\vec{\phi}$: ($J_1$ equations with $J_2$ unknowns)
%\[\tilde{L}\theta = \log \vec{\phi}\]
%In this case, we can first find $\Theta$ as in last section and then solve the constrained Least Square problem to get the QE parameter estimates $\theta$:
%\begin{align*}
%\text{minimize } & ||\theta - \Theta||^2_2 \\
%\text{subject to } & B e^{\tilde{L}\theta} =  \vec{\pi}/\pi(0)\\ 
%\text{and } & C e^{\tilde{L}\theta} =  \vec{\mu}/\pi(0)
%\end{align*} 

\newpage

\subsubsection{Sampling Algorithm}
\begin{algorithm}[H]
\caption{Usample(n, $\mu, \pi_0, \pi_{0-}$)}
\begin{algorithmic}[1]
\If {neither $\mu, \pi_{0-}$ is NULL}
\State sample $\phi$ uniformly from feasible region defined by 
\begin{align*}
\phi > & 0\\
B \phi = & \pi_{0-}/\pi_0\\ 
C \phi = & \mu/\pi_0
\end{align*}
\ElsIf {only $\mu$ is NULL}
\State sample $\phi$ uniformly from feasible region defined by 
\begin{align*}
\phi > & 0\\
B \phi = & \pi_{0-}/\pi_0
\end{align*}
\Else
\State sample $\phi$ uniformly from feasible region defined by 
\begin{align*}
\phi > & 0\\
C \phi = & \mu/\pi_0
\end{align*}
\EndIf
\State {\bf return} $n$ samples of $\phi$
\end{algorithmic}
\end{algorithm}


\begin{algorithm}[H]
\caption{SampleByBlock($n_{Iter}, n_{Burn}, L, \mu_{init}, \pi_{init}, \sigma$)}
\begin{algorithmic}[1]
\State Initialize $\mu^{(0)} = \mu_{init}$, $\pi^{(0)} = \pi_{init}$
\For {$t \in 1 \text{ to } (n_{Iter} + n_{Burn})$}
\State $\phi = \text{Usample} (1, \mu=\text{NULL}, (\pi_0,\pi_{0-})=\pi^{(t-1)})$
\State $\mu^* = \pi_0^{(t-1)} C \phi $
\State $\alpha = \frac{P(L, \mu^*, \pi^{(t-1)} )}{P(L, \mu^{(t-1)}, \pi^{(t-1)} )}$
\State $\mu^{(t)} = \mu^*$ with probability $\max(1, \alpha)$; other wise, $\mu^{(t)} = \mu^{(t-1)}$
\State $\pi^*_{0} = \text{logit}^{-1}[logit(pi_{0}^{(t-1)}) + \epsilon]$, where $\epsilon \sim $ N$(0, \sigma)$.
\State $\phi = \text{Usample} (1, \mu=\mu^{(t)}, \pi_0=\pi^*_{0},\pi_{0-}=\text{NULL})$
\State $\pi^*_{0-} = \pi^*_{0} B \phi$
\State $\pi^* = (\pi^*_{0}, \pi^*_{0-})$
\State $\alpha = \frac{P(L, \mu^{(t)}, \pi^* )}{P(L, \mu^{(t)}, \pi^{(t-1)} )}$
\State $\pi^{(t)} = \pi^*$ with probability $\max(1, \alpha)$; other wise, $\pi^{(t)} = \pi^{(t-1)} $
\EndFor
\State {\bf return} $\mu^{(n_{Burn}+1)}$ to $\mu^{(n_{Burn} + n_{Iter})}$ and $\pi^{(n_{Burn}+1)}$ to $\pi^{(n_{Burn} + n_{Iter})}$
\end{algorithmic}
\end{algorithm}



\newpage
\subsection{Measurement Error Model}
Let $M$ be the observed measurement, and $L$ be the latent true state, with $\gamma$ and $\delta$ representing the TPR and FPR.
\begin{align*}
P(\mu, \pi , \gamma, \delta|M) \propto & P(M, \mu, \pi, \gamma, \delta) \\
= & P(M | \mu, \pi, \gamma, \delta) P(\mu, \pi, \gamma, \delta) \\
= & \sum_{L \in \mathbb{L}}  P(M, L | \mu, \pi, \gamma, \delta) P(\mu, \pi, \gamma, \delta) \\ 
= & \sum_{L \in \mathbb{L}} \big[ P(M | L, \gamma, \delta) P(L | \mu, \pi) \big ] P(\mu, \pi) P(\gamma) P(\delta) \\ 
\end{align*}
where 
\begin{align*}
 P(M | L, \gamma, \delta) = & \prod_{k=1}^K P(M_k | L_k, \gamma, \delta) \text{\ \ (conditional independence assumption)}\\
 = & \prod_{k=1}^K (\gamma^{L_k} \delta^{1-L_k})^{M_k} [(1-\gamma)^{L_k} (1-\delta)^{1-L_k}]^{1-M_k}
\end{align*}
$P(\gamma)$ and $ P(\delta)$ are beta priors, and $P(L | \mu, \pi)  P(\mu, \pi)$ is defined in the same way as in previous section.


\subsubsection{Sampling Algorithm}

%\begin{table}[htbp]
%\begin{adjustwidth}{-0.5in}{-1in}
%\resizebox{0.7\textwidth}{!}{\begin{minipage}{\textwidth}
%\caption{Summary of the model coefficients and standard error estimates}
%\label{tab: coef}
%
%\begin{tabular}{llllll|lllll} 
%\hline 
% \\
%\hline \\
%\end{tabular}
%\end{minipage}}
%\end{adjustwidth}
%\end{table}





\newpage
\section*{Reference}
1. Smith RL . “Efficient Monte-Carlo Procedures for Generating Points Uniformly Dis- tributed over Bounded Regions.” Operations Research, 32(6), 1296–1308 (1984).\\ 
2. Van den Meersche, Karel, K. E. R. Soetaert, and D. J. Van Oevelen. "xsample (): an R function for sampling linear inverse problems." Journal of Statistical Software 30 (2009).


\end{document}






















